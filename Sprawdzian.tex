\documentclass[12pt,a4paper]{article}

\input{/home/tabaka/Documents/GitHub/Defs/defs.tex}

%\pagestyle{empty} 			%usuwa nr strony

\begin{document}
	
	\begin{center}
		\LARGE Sprawdzian z matematycznych aspektów wyborów
		
		Rok akademicki 2024/2025
	\end{center}
	
	\begin{center}
		\large Część teoretyczna
	\end{center}
	
	\begin{enumerate}[1.]
		\item Napisać, co to znaczy, że metoda zwycięzcy jest prawie decyzyjna. (2 pkt)
		\item Napisać, co to znaczy, że metoda zakładająca uporządkowanie spełnia kryterium przegranych Condorceta (wyjaśniając także pojęcie przegranego Condorceta). W przypadku używania oznaczeń należy je objaśnić. (2 pkt)
		\item Napisać, co to znaczy, że metoda rozdziału spełnia warunek populacji. W przypadku używania oznaczeń należy je objaśnić. (2 pkt)
		
		\newpage
		
		\begin{center}
			\large Część praktyczna
		\end{center}
		
		\item Rozważamy metodę, w której każdy z wyborców ustawia kandydatów według swojej kolejności; za pierwsze miejsce na liście kandydat dostaje 2 punkty, za drugie - 1 punkt, za trzecie również 1 punkt, za każde następne - 0 punktów. Zwycięzcą zostaje kandydat (kandydaci) z największą liczbą punktów. Zbadać (z uzasadnieniem, tzn. wskazując lub pokazując kontrprzykłady) czy metoda:
		\begin{itemize}
			\item jest efektywna
			\item jest decyzyjna
			\item jest monotoniczna ze względu na transpozycje
			\item jest jednoznacznie większościowa
			\item spełnia kryterium Condorceta
			\item spełnia słaby warunek IIA (niezależność porażki od ubocznych opcji)
		\end{itemize}
		
		\textit{Uwaga:} uzasadnienie może (a nawet powinno) być \textbf{krótkie}, czasami wystarczyć może wręcz jednozdaniowe; ważne, by zawierało to, co jest dla uzasadnienia istotne, a kontrprzykład wystarczy podać, nie trzeba pisać uzasadnienia, dlaczego jest poprawny - jednak przy podaniu kontrprzykładu należy zaznaczyć, których kandydatów on dotyczy.
		
		(1/1/1/2/3/3 pkt.)
		
		\item Rozważamy porządkową metodę głosowania, w której każdy wyborca ustawia kandydatów według swojej kolejności. O ostatecznej kolejności decyduje:
		\begin{itemize}
			\item większa liczba pierwszych miejsc na listach wyborców,
			\item w przypadku równej liczby pierwszych miejsc na listach – mniejsza liczba ostatnich miejsc,
			\item jeśli obie te liczby są równe, kandydaci są umieszczeni w końcowym wyniku na tej samej pozycji.
		\end{itemize}
		Zbadać (wykazać lub podać kontrprzykład; podobnie jak w zadaniu poprzednim, nie pisać zbyt dużo), czy ta metoda:
		\begin{itemize}
			\item spełnia porządkową zasadę Pareto,
			\item spełnia porządkowy warunek IIA,
			\item jest odporna na nieobecność.
		\end{itemize}
		
		\textit{Uwaga:} podobnie jak w zadaniu poprzednim – należy wykazać lub podać kontrprzykład i nie pisać zbyt dużo.
		
		(2/2/2 pkt.)
		
		
		\item Wykazać, że jeśli – dla dowolnego $n \geq 2$ – dwanaście osób głosuje metodą punktów Bordy na $n$ kandydatów, to nie może się zdarzyć, że kandydat, który przez siedmiu wyborców jest postawiony na ostatnim miejscu, zostanie zwycięzcą.
		
		(4 pkt.)
		
		\item Wykazać, że zbiór $\{A \subset \mathbb{N} : \mathbb{N} \setminus A \text{ jest skończony}\}$ jest filtrem na $\mathbb{N}$, ale nie jest ultrafiltrem. 
		
		(4 pkt.)
		
		\item Rozważamy porządkową metodę głosowania, w której liczba wyborców jest o jeden mniejsza niż liczba kandydatów. Wyborcy są uporządkowani: $w_1, w_2, \dots, w_{n-1}$. Wyborca $w_1$ decyduje o pierwszym miejscu, stawiając tam kandydata z pierwszego miejsca swojej listy. Wyborca $w_2$ stawia na drugim miejscu kandydata, który jest najwyżej na jego liście po usunięciu kandydata wybranego przez $w_1$; wyborca $w_3$ stawia na trzecim miejscu kandydata, który na jego liście jest najwyżej spośród tych, którzy jeszcze nie mają przyporządkowanego ostatecznego miejsca itd.; wyborca $w_{n-1}$ „obsadza” przedostatnie miejsce, a na ostatnim umieszczany jest kandydat przez nikogo nie wskazany. Zbadać (wykazać lub obalić), czy ta metoda spełnia postulat liberalizmu Sena. 
		
		(4 pkt.)
		
		\item Władzę w Dyrdymale sprawuje rada, składająca się z Burmistrza Grubego, Burmistrza Chudego oraz z trzech radnych. Wniosek (głosowany metodą TAK/NIE) przechodzi, gdy spełniony jest któryś z warunków:
		\begin{itemize}
			\item popierają go obaj burmistrzowie,
			\item popiera go jeden z burmistrzów i wszyscy trzej radni.
		\end{itemize}
		Obliczyć indeks Penrose’a–Banzhafa dla Burmistrza Grubego. 
		
		(3 pkt.)
		
		\item Herr Flick z Gestapo zaaresztował dwóch oficerów niemieckich: pułkownika von Strohma i porucznika Grubera oraz cztery osoby mieszkające w Caf\'{e} Ren\'{e}: Ren\'{e} Artois, jego żonę Edith i dwie kelnerki – Ivette i Mimi. Szóstka uwięzionych zastanawiała się nad próbą ucieczki. Uznano, że o tym, czy podejmą tę próbę, zdecyduje głosowanie, przy czym próba zostanie podjęta, jeśli w grupie głosujących za ucieczką znajdą się:
		\begin{itemize}
			\item obaj oficerowie,
			
			lub
			\item jeden oficer i jeden mieszkaniec kawiarni.
		\end{itemize}
		Obliczyć indeks Shapleya–Shubika dla Ren\'{e}. 
		
		(3 pkt.)
		
		\item Dzielimy akcje między akcjonariuszy według następującej zasady: każdemu akcjonariuszowi przyznajemy górną quotę pomniejszoną o 1, a pozostałe akcje przyznajemy tym akcjonariuszom, którzy uzyskali największą różnicę między quotą a dolną quotą (w przypadku równych różnic pierwszeństwo ma akcjonariusz z większym wkładem) – do wyczerpania akcji do rozdziału. Zakładamy, że wszyscy akcjonariusze włożyli parami różne wkłady i każdy włożył niezerowy wkład. Pokazać na przykładach, że metoda ta nie spełnia:
		\begin{itemize}
			\item warunku quoty,
			\item warunku monotoniczności.
		\end{itemize}
		W przypadku podawania przykładu należy – w szczególności – podać wkłady wyrażone w liczbach oraz liczbę akcji do rozdziału.
		
		\textit{Uwaga:} istnieją kontrprzykłady, w których występują naprawdę duże liczby.
		
		(2/2 pkt.)
		
	\end{enumerate}
		
\end{document}