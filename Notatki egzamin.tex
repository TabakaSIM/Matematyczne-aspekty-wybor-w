\documentclass[12pt,a4paper]{article}
\usepackage[utf8]{inputenc} %polskie znaki
\usepackage[T1]{fontenc}	%polskie znaki
\usepackage{amsmath}		%matematyczne znaczki :3
\usepackage{enumerate}		%Dodatkowe opcje do funkcji enumerate
\usepackage{geometry} 		%Ustawianie marginesow
\usepackage{graphicx}		%Grafika
\usepackage{wrapfig}		%Grafika obok textu
\usepackage{float}			%Allows H in figure
\usepackage{hyperref}		%Allows hyperlinks
%\pagestyle{empty} 			%usuwa nr strony
\usepackage{todonotes}		%Todo notatki
\usepackage{lipsum}         %Lorem text
\usepackage{ntheorem}   	% for theorem-like environments
\usepackage{mdframed}   	% for framing
\usepackage{subcaption}		% subfigure (image placing)
\usepackage{pdfcomment}		% Komentarze (z bazowego pdf'a)
\usepackage{xparse}			% New commands with optional arguments
\usepackage{ifthen}			% If then - funkcje!
\usepackage{expl3}			% Deklarowanie zmiennych
\usepackage{pgf}			% Aktualne rachunki \pgfmathparse{}
\usepackage{amsmath} 		% For mathematical symbols and structures
\usepackage{amsfonts}		% Zbiór liczb naturalnych + formatowanie
\usepackage{ulem}			% Przekreślony text
%\usepackage[colorlinks=true, linkcolor=blue, urlcolor=red, citecolor=green]{hyperref}
\usepackage{fontawesome5}
\usepackage{mathtools}
%\usepackage{multirow} 		% Required for merging rows

\newgeometry{tmargin=2cm, bmargin=2cm, lmargin=2cm, rmargin=2cm} 

%Counter commands{
	\newcounter{definicja}
	\setcounter{definicja}{1} 
	\newcounter{twierdzenie}
	\setcounter{twierdzenie}{1} 
	\newcounter{przyklady}
	\setcounter{przyklady}{1} 
	\newcounter{wnioski}
	\setcounter{wnioski}{1} 
	
	\newcommand{\counter}[1]{
		\arabic{#1} \stepcounter{#1} 
	}
	\newcommand{\counterreset}[1]{\setcounter{#1}{1}}
	%}

%Define styles{
	\theoremstyle{break}
	\theoreminframepreskip{0.5cm}
	\theoremheaderfont{\bfseries}
	\newmdtheoremenv[%
	linecolor=white,%
	innertopmargin=\topskip,
	shadowsize=0,%
	innertopmargin=5,%
	innerbottommargin=5,%
	leftmargin=10,%
	rightmargin=10,%
	backgroundcolor=gray!20,%
	innertopmargin=0pt,%
	ntheorem]{zad}{Zadanie}
	
	\mdfdefinestyle{zadanie}{
		linecolor=white,%
		innertopmargin=5,%
		innerbottommargin=5,%
		leftmargin=10,%
		rightmargin=10,%
		backgroundcolor=gray!20,%
		innertopmargin=8,
		innerbottommargin=8,
		skipabove = 5,
	}
	\mdfdefinestyle{wzor}{
		linecolor=cyan,%
		linewidth=2pt,%
		innertopmargin=8,
		innerbottommargin=8,
		leftmargin=10,%
		rightmargin=10,%
		backgroundcolor = white, 
		fontcolor = black,
		skipabove = 5,
		skipbelow = 5,
	}
	%}

%Zadania templatex%{
	\newcommand{\Obramowka}[1]{
		\begin{mdframed}[style=wzor]
			\centering #1
		\end{mdframed}
	}
	\newcommand{\Komentarz}[1]{
		\begin{mdframed}[style=zadanie]
			\textbf{Komentarz}\\
			#1
		\end{mdframed}
	}
	
	%}

% Set spacing before and after theorems
\setlength{\theorempreskipamount}{20pt}  % Space above the theorem
\setlength{\theorempostskipamount}{20pt} % Space below the theorem

\newtheorem{definition}{Definicja}[section]

\newtheorem{theorem}{Twierdzenie}[section]
\newtheorem{lemma}{Lemat}[section]
\newtheorem{wniosek}{Wniosek}[theorem]
\newtheorem{example}{Przykład}[section]
\newtheorem{exercise}{Ćwiczenie}[section]
\newtheorem{stwierdzenie}{Stwierdzenie}[section]
\newtheorem{obserwacja}{Obserwacja}[section]


\newcommand{\tg}{\text{tg}}
\newcommand{\ctg}{\text{ctg}}
\newcommand{\witw}{$\Leftrightarrow$}
\newcommand{\wynika}{$\Rightarrow$}
\newcommand{\UkladRownan}[2]{
	$\left\{
	\begin{array}{l}
		#1 \\
		#2
	\end{array}
	\right.$
}
\begin{document}
	\section{Notatnki}
	\begin{definition}[Metoda anonimowa]
		Metoda jest anonimowa \witw gdy wszyscy wyborcy są traktowani tak samo, to znaczy \witw $\forall_{x,y \in W}$ zamiana głosów $x$ i $y$ nie zmienia wyniku.
	\end{definition}
	
	\begin{definition}[Metoda neutralna]
		Metoda jest neutralna \witw wszyscy kandydaci są traktowani tak samo, to znaczy \witw $\forall_{x,y \in K}$ zamiana ról $x$ i $y$ nie zmienia wyniku.
	\end{definition}

	\begin{definition}[Metoda efektywna]
		Metoda zwycięzcy jest efektywna \witw zawsze wyłania przynajmniej jednego zwycięzcę.
	\end{definition}

	\begin{definition}[Metoda decyzyjna]
		Metoda zwycięzcy jest decyzyjna \witw w każdym modelu wyłania dokładnie jednego zwycięzcę.
	\end{definition}
	
	\begin{definition}[Metoda prawie decyzyjna]
		Metoda zwycięzcy jest prawie decyzyjna \witw w każdym modelu wyłania co najwyżej jednego zwycięzcę. Sytuacja, w której nie ma zwycięzcy, zachodzi wtedy, gdy więcej niż jeden kandydat uzyskał tę samą, najwyższą liczbę punktów.
	\end{definition}

	\begin{definition}[Kryterium jednoznacznej bezwzględnej większości]
		Metoda zwycięzcy (MZ) spełnia kryterium jednoznacznej bezwzględnej większości, \textit{wtedy i tylko wtedy}, gdy kandydat, który otrzyma ponad 50\% głosów, jest jedynym zwycięzcą.
	\end{definition}

	\begin{definition}[Metoda monotoniczna ze względu na zwycięzcę]
		\textbf{Z:} MZ - klasyczna lub semi-klasyczna.
		Metoda jest monotoniczna ze względu na zwycięzcę, \textit{wtedy i tylko wtedy}, gdy kandydat $A$ jest zwycięzcą, a jeśli wybierzemy kandydata $B$ różnego od $A$ ($B\neq A$) oraz jego grupę wyborców, to jeśli ta grupa zmieni swoje głosy bez straty dla $A$ (czyli zmiana nastąpi zgodnie z następującymi dozwolonymi operacjami):\\
		
		\begin{tabular}{|c|c|c|}
			\hline
			M & $\Rightarrow$ & N \\\hline
			A B & $\Rightarrow$ & A B \\\hline
			- - & $\Rightarrow$ & - - \\\hline
			+ + & $\Rightarrow$ & + + \\\hline
			- + & $\Rightarrow$ & + - \\\hline
		\end{tabular}\\
		
		to: $\Rightarrow$ $A$ nadal wygrywa.
	\end{definition}

	\begin{theorem}[Maya - \textit{Kenneth Maya, 1952r}]
		\textbf{Z:} Klasyczna metoda zwycięzcy oraz $\# K = 2$. Jeżeli metoda ta jest metodą:
		\begin{enumerate}[(1)]
			\item anonimową
			\item neutralną
			\item monotoniczną ze względu na zwycięzcę
			\item prawie decyzyjną
		\end{enumerate}
		$\Rightarrow$ jest metodą bezwzględnej większości.
	\end{theorem}

	\begin{definition}[Metoda zakładająca uporządkowanie]
		Metoda zwycięzcy jest metodą zakładającą uporządkowanie (MZU) \witw gdy $\forall_{w\in W}$ wyborca $w$ ustala $K$ kandydatów w liniowym porządku, a jego głos zależy od tego porządku.
	\end{definition}

	\begin{definition}[Monotoniczność ze względu na transpozycję]
		MZU jest monotoniczna ze względu na transpozycje \witw $\forall_{M-\text{model}} \forall_{w\in W} \forall_{A,B\in K}$, jeśli w $M$ głosuje się $[\Delta, B, A, *]$ i w $M$ wygrywa $A$, to w $N$, gdzie zmiana polega na $[\Delta, A, B, *]$, również wygrywa $A$.
	\end{definition}

	\begin{definition}[Słaba zasada Pareto]
		MZU spełnia słabą zasadę Pareto \witw $\forall_M (\exists_{A,B\in K}\forall_{w\in W} A\overset{w,M}{<}B) \Rightarrow A$ nie wygrywa w $M$.
	\end{definition}
	
	\begin{definition}[Kandydat Condorceta]
		$A\in K$ jest kandydatem Condorceta (zwycięzcą Condorceta) \witw w "bezpośrednich porównaniach" $A$ jest lepszy od każdego innego kandydata \witw $\forall_{B \in K}: B\neq A \quad \#\{w: B\overset{w,M}{<}A\}>\#\{w: B\overset{w,M}{>}A\}$.
	\end{definition}

	\begin{definition}[Przegrany Condorceta]
		$A\in K$ to przegrany Condorceta (w modelu $M$) \witw $\forall_{B \in K}: B\neq A \quad\#\{w: B\overset{w,M}{<}A\}<\#\{w: B\overset{w,M}{>}A\}$.
	\end{definition}
	
	\begin{definition}[Kryterium Condorceta]
		Metoda spełnia kryterium Condorceta \witw $\forall_M:$ Istnieje kandydat Condorceta (w $M$) $\Rightarrow A$ - jedyny zwycięzca w $M$.
	\end{definition}
	
	\begin{definition}[Kryterium przegranych Condorceta]			
		Metoda spełnia kryterium przegranych Condorceta \witw $\forall_M:$ Istnieje przegrany Condorceta $A$ w $M$ $\Rightarrow A$ nie wygrywa w $M$.
	\end{definition}
	
	\begin{definition}[Metoda jednoznacznie większościowa]
		Metoda jest jednoznacznie większościowa \witw $\forall_M$ kandydat $A$ w $M$ ma ponad połowę pierwszych miejsc $\Rightarrow A$ - jedyny zwycięzca.
	\end{definition}
	
	\begin{definition}[Metoda słabo niezależna od ubocznych opcji IIA]
		Metoda słabo niezależna od ubocznych opcji spełnia warunek niezależności porażki od ubocznych opcji (spełnia słaby warunek $IIA$) \witw
		
		$\forall_{A,B\in K} \forall_{M,N - modele}$ spełnia 
		$\begin{bmatrix}
			\forall_{w\in W} (B\overset{w,M}{<}A) \text{\witw} (B\overset{w,N}{<}A) \\
			\text{A wygrywa w M, B nie wygrywa w M}		 
		\end{bmatrix}$	$\Rightarrow$ B nie wygrywa w N.\\\\
		
		Innymi słowy: zmiany nie wpływające na relacje przegrany-zwycięzca nie mogą dać przegrywającemu zwycięstwa.		
		
	\end{definition}

	\begin{theorem}
		MZU, anonimowa, neutralna $\Rightarrow$ metoda nie jest decyzyjna.
	\end{theorem}

	\begin{theorem}
		MZU spełnia kryterium Condorceta $\Rightarrow$ jest jednoznacznie większościowa.
	\end{theorem}

	\begin{lemma}[Lemat o decyzyjności]
		Z: MZU, efektywna, $\#K\geq3$, $\#W\geq2$.
		Metoda spełnia słabą zasadę Pareto i słabe IIA
		$\Rightarrow$ jest decyzyjna.
	\end{lemma} 

	\begin{theorem}[Twierdzenie Arrowa dla metod zwycięzcy (1951)]
		Załóżmy: $\# K \geq 3, \#W\geq2$, MZU, efektywna, spełnia słabą zasadę Pareto i słabe IIA $\Rightarrow$ dyktatura.
	\end{theorem}

	\begin{wniosek}[Twierdzenie Arrowa o niemożliwości]
		Załóżmy, że $\# K \geq 3$, $\# W \geq 2$. Wówczas: nie istnieje metoda zakładająca uporządkowanie (MZU) efektywna, która jednocześnie spełnia następujące warunki:
		\begin{enumerate}[-]
			\item anonimowa,
			\item słabą zasadę Pareto,
			\item słabe kryterium niezależności od opcji ubocznych (IIA).
		\end{enumerate}
	\end{wniosek}

	\begin{theorem}[Twierdzenie Taylora o niemożliwości]
		Dla $\# K \geq 3$, $\# W \geq 3$:  
		nie istnieje MZU efektywna, która spełnia jednocześnie kryterium Condorceta oraz słabe kryterium niezależności od ubocznych opcji (IIA).
	\end{theorem}

	$$
	\begin{bmatrix}
		\Sigma = \{ M: W \rightarrow K \} & \Sigma = \{ M: W \rightarrow \{ \text{TAK}, \text{NIE} \} \} \\
		f:\Sigma \rightarrow P(K) & f:\Sigma \rightarrow \{ \text{TAK}, \text{NIE} \}
	\end{bmatrix}
	$$
	
	\begin{definition}[Założenia metody TAK/NIE]
		Założenia:´
		
		\begin{itemize}
			\item $\forall_{w \in W} \; w : \text{TAK} \Rightarrow \text{wynik: TAK}$,
			\item $\forall_{w \in W} \; w : \text{NIE} \Rightarrow \text{wynik: NIE}$.
		\end{itemize}
	\end{definition}

	\begin{definition}[Koalicja wygrywająca]
		Podzbiór $A \subset W$ jest koalicją wygrywającą wtedy i tylko wtedy, gdy:  
		$$\{x \in A : x \text{ głosuje TAK} \} \Rightarrow \text{wynik: TAK.}$$
	\end{definition}
	
	\begin{definition}[Monotoniczność metody TAK/NIE]
		Metoda TAK/NIE jest monotoniczna wtedy i tylko wtedy, gdy:  
		$$
		\begin{bmatrix}
			A \subset A_1 \\
			A \text{ jest koalicją wygrywającą}
		\end{bmatrix}
		\Rightarrow A_1 \text{ jest koalicją wygrywającą.}
		$$
	\end{definition}
	
	\begin{definition}[Wskaźnik Banzhafa]
		Załóżmy, że $W = \{a_1, \dots, a_n\}$.  
		Wskaźnik Banzhafa dla $a_i$ jest równy liczbie:
		$$B(a_i) = \# \{ A : a_i \in A \; \text{i $a_i$ jest decydujący dla $A$} \}.$$
	\end{definition}
	
	\begin{definition}[Indeks Banzhafa (Penrose’a–Banzhafa)]
		Indeks Banzhafa dla $a_i$ definiuje się jako:
		$$I_B(a_i) = \frac{B(a_i)}{B(a_1) + \dots + B(a_n)}.$$
	\end{definition}

	\begin{definition}[Wskaźnik/Indeks Shapleya–Shubika]
		Dla metody monotonicznej:  
		Porządkujemy wyborców jako $W = (w_1, \dots, w_n)$.  
		W ciągu $(w_1, \dots, w_n)$ wyborca $w_k$ jest wpływającym wyborcą, jeśli:
		\[
		\{w_1, \dots, w_{k-1}\} \text{ nie tworzy koalicji wygrywającej, a } \{w_1, \dots, w_k\} \text{ już tak.}
		\]
		
		Wskaźnik Shapleya–Shubika:
		\[
		S(w_k) = \# \{\text{ciągi, w których } w_k \text{ jest wpływającym wyborcą}\}.
		\]
		
		Indeks Shapleya–Shubika:
		\[
		I_S(w_k) = \frac{S(w_k)}{n!}.
		\]
	\end{definition}

	\begin{definition}[Słaby porządek]
		Zbiór $K$ jest słabo uporządkowany wtedy i tylko wtedy, gdy $\exists R$ — relacja równoważności w $K$ taka, że $K / R$ jest uporządkowany liniowo przez relację $\leq$. 
		
		Dla $a, b \in K$ definiujemy: 
		\[
		a < b \overset{\text{def}}{\Leftrightarrow} [a]_R < [b]_R,
		\]
		gdzie relacja jest przechodnia i słabo antysymetryczna.
	\end{definition}

	\begin{definition}[Metoda porządkowa (MP)]
		Każdy wyborca porządkuje kandydatów w sposób liniowy.
		
		Wynik wyborów jest słabym porządkiem w zbiorze $K$:
		\begin{itemize}
			\item $L(K) = \{(K, \leq) : \leq \text{ jest porządkiem w } K\}$,
			\item $S(K) = \{(K, \leq) : \leq \text{ jest słabym porządkiem w } K\}$.
		\end{itemize}
		
		Zapis formalny:
		\[
		\Sigma = \{ M : W \rightarrow L(K) \}, \quad f : \Sigma \rightarrow S(K).
		\]
	\end{definition}

	\begin{definition}[Porządkowa zasada Pareto]
		Metoda porządkowa (MP) spełnia (porządkową) zasadę Pareto wtedy i tylko wtedy, gdy:
		\[
		\forall_M \forall_{A, B \in K} \left( \forall_w \; A \overset{w, M}{<} B \right) \Rightarrow A \underset{M}{<} B.
		\]
	\end{definition}
	
	\begin{definition}[Metoda spełniająca postulat liberalizmu Sena]
		Metoda spełnia postulat liberalizmu Sena, jeśli:
		\[
		\forall_{w \in W} \exists_{A, B \in K \; (A \neq B)} \; \left( A \overset{w, M}{<} B \Rightarrow A \underset{M}{<} B \right) \text{ oraz } \left( A \overset{w, M}{>} B \Rightarrow A \underset{M}{>} B \right).
		\]
	\end{definition}
	
	\begin{theorem}[Twierdzenie Sena]
		Niech $\# K \geq 2$ oraz $\# W \geq 2$. Wtedy:
		Nie istnieje metoda spełniająca jednocześnie:
		\begin{enumerate}
			\item zasadę Pareto,
			\item postulat liberalizmu Sena.
		\end{enumerate}
	\end{theorem}

	\begin{definition}[Filtr]
		Niech $X$ będzie zbiorem. Podzbiór $F \subset P(X)$, gdzie $F \neq \emptyset$, nazywamy **filtrem**, jeśli spełnia następujące warunki:
		\begin{enumerate}[1)]
			\item $\emptyset \notin F$
			\item Jeśli $A, B \in F$, to $A \cap B \in F$
			\item Jeśli $A \in F$ oraz $A \subset B \subset X$, to $B \in F$
		\end{enumerate}
	\end{definition}

	\begin{definition}[Ultrafiltr]
		Podzbiór $F \subset P(X)$ nazywamy **ultrafiltrem**, jeśli spełnia następujące warunki:
		\begin{enumerate}[1)]
			\item $F$ jest filtrem.
			\item Dla każdego $A \subset X$ zachodzi dokładnie jedno z dwóch: $A \in F$ albo $X \setminus A \in F$.
		\end{enumerate}
	\end{definition}

	\begin{definition}[Liberalizm Senna]
		\[
		\forall_{w \in W} \exists_{a, b \in K; a \neq b} \forall_M \; w \text{ decyduje o } (a, b)
		\]
	\end{definition}

	\begin{theorem}
		Dla $K$ i $W$ istnieje metoda spełniająca postulat liberalizmu Senna \textit{wtedy i tylko wtedy}, gdy $\# W < \# K$.
	\end{theorem}

	\begin{definition}[Warunki sensowności metody rozdziału]
		\begin{enumerate}[1.]
			\item (Warunek quoty) $\lfloor q_i \rfloor \leq a_i \leq \lceil q_i \rceil$ (uwzględniając $q_i \in \mathbb{Z} \Rightarrow g_i = a_i$)
			\item (Warunek monotoniczności) $p_i > p_j \Rightarrow a_i \geq a_j$ (analogicznie w drugą stronę)
			\item (Warunek populacji) Dla $S, m$ danych:
			\[
			p_1, \dots, p_s \longmapsto a_1, \dots, a_s
			\]
			jeśli nastąpiła zmiana:
			\[
			\bar{p_1}, \dots, \bar{p_s} \longrightarrow \bar{a_1}, \dots, \bar{a_s},
			\]
			to:
			\[
			!\exists_{i,j} \bar{p_i}>p_i, \bar{a_i}<a_i \quad \text{oraz} \quad \bar{p_j}<p_j, \bar{a_j}>a_j
			\]
			\item (Warunek monotoniczności akcji) Dla $p_1,\dots,p_n$ stałych, jeśli $\bar{m}>m$, to $\forall_i \bar{a_i} \geq a_i$.
		\end{enumerate}
		Nie mogą zajść wszystkie te warunki na raz.
	\end{definition}

		\textbf{Paradoks Alabamy}\\
	Dla $S=3$, $m=10$, $W=100$, po zmianie liczby akcji na $m=11$, $W=90,9$, wyniki się zmieniają:
	\[
	\begin{array}{|c|c|c|c|}
		\hline
		p_i & q_i & \lfloor q_i \rfloor & \text{Wynik} \\\hline
		145 & 1,45 & 1 & 2 \\
		340 & 3,40 & 3 & 3 \\
		515 & 5,15 & 5 & 5 \\\hline
		\Sigma=1000 & & 9 & 10 \\\hline
	\end{array}\Rightarrow\begin{array}{|c|c|c|c|}
		\hline
		p_i & q_i & \lfloor q_i \rfloor & \text{Wynik} \\\hline
		145 & 1,595 & 1 & 1 \\
		340 & 3,740 & 3 & 4 \\
		515 & 5,665 & 5 & 6 \\\hline
		\Sigma=1000 & & 9 & 11 \\\hline
	\end{array}
	\]
	
	\textbf{Paradoks Oklahomy}\\
	Po zmianie liczby akcjonariuszy i akcji, np. $S=4,\: m=13, \: W=76,9$, także mogą wystąpić sprzeczne wyniki.
	
	\begin{theorem}[Tw. Balińskiego-Younga]
		Dla $S \geq 4$ i $m \geq 7$ nie istnieje metoda spełniająca jednocześnie warunki:
		\begin{itemize}
			\item quoty,
			\item monotoniczności,
			\item populacji.
		\end{itemize}
	\end{theorem}

	\begin{definition}[Metoda dzielników]
		Metoda rozdziału nazywana jest metodą dzielników, jeśli istnieje funkcja 
		$f: [0, \infty) \to \mathbb{N}$, taka że:
		\begin{enumerate}[a)]
			\item $x \in \mathbb{Z} \Rightarrow f(x) = x$
			\item Funkcja jest rosnąca: $x \leq y \Rightarrow f(x) \leq f(y)$
		\end{enumerate}
	\end{definition}

	\begin{definition}[Metoda wartościująca -- XXI wiek]
		Niech $S$ będzie zbiorem stopni (ocen wartości). Metoda $M: W \to \{ f: K \to S \}$ polega na tym, że wyborca każdemu kandydatowi przypisuje ocenę.
	\end{definition}

	\begin{definition}[Metoda rankingowo niezależna od ubocznych opcji]
		Metoda wartościująca jest rankingowo niezależna od ubocznych opcji wtedy i tylko wtedy, gdy:
		\[
		\forall_{ M, N} \, \forall_ {A, B} \, 
		\begin{bmatrix}
			M, N \text{ -- modele} \\
			K_M = K_N \cup \{C\}, \, C \notin K_M \\
			\forall v \in W \, \text{ oceny } v \text{ w } M = \text{ oceny } v \text{ w } N
		\end{bmatrix}
		\Rightarrow 
		(A \underset{M}{<} B \, \Leftrightarrow \, A \underset{N}{<} B).
		\]
	\end{definition}

		\begin{definition}[Metoda odporna na nieobecność]
		Metoda (MP, MW) jest odporna na nieobecność wtedy i tylko wtedy, gdy:
		\[
		\forall_{M,N} \forall_{A,B} 
		\begin{bmatrix}
			M, N \text{ -- modele} \\
			W_N = W_M \cup W^*, \quad W^* \cap W_M = \emptyset \\
			v \in W_M \Rightarrow \text{ głos } v \text{ w } M = \text{ głos } v \text{ w } N \\
			A \underset{M}{<} B \quad \text{oraz} \quad \forall_{v \in W^*} A \overset{v,N}{<}B
		\end{bmatrix}
		\Rightarrow A \underset{N}{<} B.
		\]
	\end{definition}
\end{document}